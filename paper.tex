% Template for ICIP-2022 paper; to be used with:
%          spconf.sty  - ICASSP/ICIP LaTeX style file, and
%          IEEEbib.bst - IEEE bibliography style file.
% --------------------------------------------------------------------------
\documentclass{article}

\usepackage{spconf}
\usepackage{times}
\usepackage{epsfig}
\usepackage{graphicx}
\usepackage{amsmath}
\usepackage{amssymb}
\usepackage{multirow}
%\usepackage{caption}
\usepackage{subcaption}
\usepackage{array}
\usepackage{xcolor}
\usepackage{ifthen}

% If you comment hyperref and then uncomment it, you should delete
% egpaper.aux before re-running latex.  (Or just hit 'q' on the first latex
% run, let it finish, and you should be clear).
\usepackage[pagebackref=true,breaklinks=true,colorlinks,bookmarks=false]{hyperref}

\newcolumntype{!}{>{\global\let\currentrowstyle\relax}}

\newcolumntype{^}{>{\currentrowstyle}}
\newcommand{\rowstyle}[1]{\gdef\currentrowstyle{#1}%
    
    #1\ignorespaces
    
}

\providecommand{\red}[1]{\textcolor{red}{#1}}

\def\debug{1}

\ifnum \debug=1
\providecommand{\lnote}[1]{\textcolor{blue}{\textbf{LR: #1}}}
\providecommand{\tnote}[1]{\textcolor{teal}{\textbf{[TF: #1]}}}
\providecommand{\vnote}[1]{\textcolor{orange}{\textbf{[VB: #1]}}}
\providecommand{\anote}[1]{\textcolor{red}{\textbf{[AP: #1]}}}
\providecommand{\mnote}[1]{\textcolor{violet}{\textbf{[MB: #1]}}}
\else
\providecommand{\lnote}[1]{}
\providecommand{\tnote}[1]{}
\providecommand{\vnote}[1]{}
\providecommand{\anote}[1]{}
\providecommand{\mnote}[1]{}
\fi

% Title.
% ------
\title{Memory-augmented Online Video Anomaly Detection}
%
% Single address.
% ---------------
\name{Leonardo Rossi, Vittorio Bernuzzi, Tomaso Fontanini, Massimo Bertozzi, Andrea Prati\thanks{This research benefits from the HPC (High Performance Computing) facility of the University of Parma, Italy.}}
%\address{}
%
% For example:
% ------------
\address{IMP Lab - Dipartimento di Ingegneria e Architettura\\
    University of Parma, Italy\\
    Emails: \{name.surname\}@unipr.it \mnote{ocio!}}

% Two addresses (uncomment and modify for two-address case).
% ----------------------------------------------------------
%\twoauthors
%  {A. Author-one, B. Author-two\sthanks{Thanks to XYZ agency for funding.}}
%	{School A-B\\
%	Department A-B\\
%	Address A-B}
%  {C. Author-three, D. Author-four\sthanks{The fourth author performed the work
%	while at ...}}
%	{School C-D\\
%	Department C-D\\
%	Address C-D}
%
\begin{document}
%\ninept
%
\maketitle
%

%%%%%%%%% ABSTRACT
\begin{abstract}
    Autonomous Vehicles (AVs) are almost certainly becoming a reality for the near future thanks to the substantial recent advances in computer vision.
To make this happen, it is of paramount importance to give to AVs the ability to understand the surrounding scene.
This paper presents a system capable to work in a real-time and online fashion, giving an immediate response to the arise of anomalies surrounding the AV, exploiting only the videos captured by a dash-mounted camera.
Our architecture, called \emph{MOVAD}, relies on two main modules: a short-term memory to extract information related to the ongoing action, implemented by a Video Swin Transformer adapted to work in an online scenario, and a long-term memory module that considers also remote past information thanks to the use of a Long-Short Term Memory (LSTM) network.
We evaluated the performance of our method on Detection of Traffic Anomaly (DoTA) dataset, a challenging collection of dash-mounted camera videos of accidents.
After an extensive ablation study, MOVAD is able to reach an AUC score of 82.05\%, surpassing the current state-of-the-art by $+2.75\%$ AUC.
Our code will be available on \url{https://github.com/IMPLabUniPr/movad/tree/icip}
\end{abstract}
\begin{keywords}
    Video Transformer, LSTM, Video Anomaly Detection
\end{keywords}

%%%%%%%%% BODY TEXT
\section{Introduction and related works}

Autonomous Vehicles (AVs) are becoming every day a reality thanks to the recent enormous scientific and technical advances.
Nevertheless, safety of AVs is still a relevant issue which can jeopardize their world-wide diffusion.
Increasing safety of AVs can be reached by providing vehicles with the ability of detecting accidents in a prompt way.
In general, all anomalous traffic situations are relevant for many automatic tasks of AVs.
Their detection provides information to avoid collisions, protect pedestrians, or re-route the current travel \cite{4298901}.
%Moreover, if a V2V (vehicles to vehicles) network infrastructure (such as VANET \cite{fatemidokht2021efficient}) is available, dissemination of useful information can provide other connected vehicles insights about the traffic status.
Among the different sensors exploited in AVs, cameras can provide rich real-time information about the scene.
%Real-time video analysis can make useful tools available for driving scene interpretation in the context of Advanced Driver Assistance Systems (ADAS)\mnote{forse la toglierei}.
Despite the gigantic advances of video analysis algorithms, anomalous situation detection in real traffic scenarios is still very challenging, due to several reasons.
First of all, detection of traffic anomalies requires real-time performance in order to allow the driver and/or the vehicle's actuators to react promptly.
Real-time performance does not only call for efficient algorithms, but also implies an online strategy, meaning that the system needs to rely only on current frames and, at most, knowing the past (or part of it), but having no idea of what will happen in the future.
Secondly, aiming at detecting anomalies in traffic scenes, a proper, formal and generally-acknowledged model of what an anomaly is should be available.
Unfortunately, this shared model does not exist and, in any case, would likely be highly subjective.
%For instance, improper crossing of driving lanes might be perceived as anomalous and dangerous if long-lasting and in crowded traffic, while being negligible in other cases.
In addition, there are plenty of possible accident classes that must be taken into account compared to normal traffic situations, and, for some of them, the number of available examples are very exiguous.
Lastly, we aim to precisely localize in time the anomaly and the definition of precise boundaries of an anomaly is even more subjective and doubtful.
Indeed, we can define an anomaly as an hazardous situation that can lead to an accident, but, since the hazardousness prior to the accident may be determined subjectively by each individual, the time interval edges for an anomaly are not really clear and this is reflected in dataset labelling.
Nevertheless, some attempts have been made to propose a deterministic method in the interest of defining an anomaly.
Fang \emph{et al.} \cite{fang2019dada} want to predict an accident likely to happen in the next 5 seconds, labelling the anomaly start from the moment in which half part of the object involved in the accident appears in the view.
Yao \emph{et al.}~\cite{yao2020when} proposed a Detection of Traffic Anomaly (DoTA) dataset that takes into account when the anomalous event starts and ends, locating spatially where all the involved entities are in each frame and the type of anomaly.
Their work formulates the anomaly start as the instant after which the accident is unavoidable (still, a subjective concept).
We have decided to take this last dataset as a benchmark for our work.

Despite these premises, there have been several previous works addressing the problem of traffic anomaly detection.
Authors in \cite{hasan2016learning} proposed a Convolutional AutoEncoder (ConvAE) trained only on normal frames with the objective of frame reconstruction.
In \cite{luo2017remembering, wang2018abnormal}, authors used Convolutional LSTM Auto-Encoder as framework to encode appearance and motion.
As noted by \cite{ramachandra2020survey}, auto-encoder-type reconstruction methods are sensitive to the amount of anomalies that occur in the scene and many times they require additional post-processing techniques.
Authors in \cite{liu2018future} proposed AnoPred, which uses a multi-task loss including image intensity, optical flow, gradient, and adversarial losses for video frame-level anomaly detection by applying a UNet to predict a future frame.
As stated in \cite{9712446}, AnoPred was thought in a video surveillance context, while videos acquired on board of a moving vehicle are more dynamic and difficult to predict.
In \cite{zhou_spatio-temporal_2022}, authors make first a coarse detection, by the encoding of the temporal features with Histogram of Optical Flow (HOF) \cite{wang2013action} and ordinal features of frames by a CNN, and then a fine detection by encoding the CNN features and spatial relationships of the objects.
In Future Object Localization (FOL)~\cite{9712446}, authors try to avoid the future prediction for the entire frame, focusing instead on tracking actors position and predicting their future locations.
Conversely to us, their method is limited by the presence of actors in the Field of View (FOV).
In addition, differently from AnoPred~\cite{liu2018future}, FOL \cite{9712446} and \cite{zhou_spatio-temporal_2022}, we only need RGB frames.
In \cite{xu2019temporal}, the TRN model couples the action detection task with both the temporal dependencies modeled by a RNN and the anticipation of the future via a temporal decoder.
In our case, we do not speculate on the future, but rely only on the present and the past.

In this paper, we describe our new architecture called \emph{MOVAD} (Memory-augmented Online Video Anomaly Detection).
In order to build a system that is capable to temporally locate anomalies in videos, belonging to a wide range of possible situations, we have introduced two main architectural contributions.
The first exploits a Video Swin Transformer \cite{liu_video_2022} (VST) model, adapted to model short-term memory in a real-time and online scenario.
This adaptation allows the system to incorporate temporal correlation and to detect online behaviors, by relying only on the current and few previous frames.
The second architectural contribution is the injection of a LSTM module inside the classifier head to model the long-term memory of the past.
This is crucial to have the network learning from the remote past situations how to better model anomalies.
An exhaustive ablation study on all the components of the architecture over the public dataset DoTA \cite{9712446} shows superior performance in terms of AUC w.r.t.~the state of the art, adding a further degree of difficulty given by the real-time and online capabilities.\mnote{e comunque parliamo di real time ma ricordo male io o non si parla di tempi di esecuzione?}\anote{Ha ragione Massimo. Verifichiamo bene}
%\section{Introduction}

%Intro
In order to make Autonomous Vehicles (AV) reliable in a real-world scenario, safety for every agent involved must be the focus of every self-driving system implementation.
This objective will be achieved once the AV has a clear understanding of the driving scene around it, focusing on the main visual cues that are needed for its correct behaviour, differentiating between normal and anomalous situations, so that it can react in real-time to take the safest decision.
It is necessary to track down all possible causes for the sake of accident avoidance, this must be done with both precision and promptness to assure the maximum reaction space.

%Task
In 2019, the European Commission and the European Council jointly approved a regulation that will require new safety measures in motor vehicles.
Starting July 2022, a device similar to "black boxes" in aircrafts will be mandatory for new cars, including the installation of so-called "Event Data Recorders" (EDR).
Driven by the idea that real-time video could be available in all cars in near future, we focused our work on video analysis captured by dash-mounted cameras, with the aiming of improving the tools for driving scene interpretation in the context of Advanced Driver Assistance Systems (ADAS).
From this perspective, a driving scenario is quite hard to model since there are many information to take into account that can be exploited to define the driving scene.
There are plenty of possible accident classes that must be taken into account compared to normal traffic situations.
In addition, each of these possibilities usually has a very short duration, in the order of a few seconds.
For these reasons, we can compare the traffic video anomaly detection task to a long-tailed distribution problem.
To further compound the difficulty, the crash prediction system aims to work in real-time and online fashion.
This means that it must be fast and with predictable response times, but also work only on current frames and, at most, knowing the past, but having no idea what happens in the future.

%Anomaly
Though we know that accidents are a consequence of an anomalous driving scenario, it is non-trivial to define precise boundaries of what a driving anomaly is.
\lnote{spiegare perchè è difficile -> sarebbe spiegato nel paragrafo successivo in cui spiego che l'intervallo di anomalia è spesso definito in via soggettiva -> direi di risolverla cosi: o mettiamo qui il motivo, o la frase successiva la possiamo iniziare con una cosa tipo "Prima di definire perchè, ...", cosi si capisce che lo andiamo a spiegare dopo. Che ne pensi?}
We can define an anomaly as an hazardous situation that can lead to an accident, but since the hazardousness prior to the accident may be determined subjectively by each individual, the time interval edges for an anomaly are not really clear and this is reflected in some dataset annotations.
Some attempts have been made to propose a deterministic method in the interest of defining an anomaly.
Yao et al. \cite{yao2019unsupervised} defines an anomaly as the window in which the accident happens, but since we want to prevent it, this might not be ideal in a prevention perspective. \lnote{pero' noi non facciamo prevention, quindi?}
Fang et al. \cite{fang2019dada} instead want to predict an accident willing to happen in the next 5 seconds labelling the anomaly start from the moment in which half part of the object involved in the accident appears in the view.
Yao et al. \cite{yao2020when} proposed a Detection of Traffic Anomaly (DoTA) dataset that takes into account when the anomalous event starts and ends, locates spatially where all the involved agents are in each frame and what type of anomaly it is.
Their work formulates the anomaly start as the instant after which the accident is unavoidable.
As said before choosing that instant is quite subjective depending on the situation and personal biases, in fact Lung et al. \cite{lund2009riskperception} argues that risk perception is related to social constructs that reflect the cultural context in which people live in.
\lnote{<- se alla fine non inseriamo le nuove annotazioni, questa parte non so se lasciarla. possiamo riscriverla in funzione del paragrafo successivo?}  

% intenzioni da cui siamo partiti
Sometimes traffic anomalies are recognizable from the single frame, such as the presence of objects in places where they shouldn't be, e.g. boulders due to a landslide or animals in the middle of the roadway.
Often, however, the anomalies are attributable to actions that last for a few seconds and, in the worst case, a few tenths of a second.
Such as for example the crossing of a scooter at the wrong time, or the loss of control of the vehicle due to the presence of snow.
To build a system that is capable of recognizing both situations, we have outlined four fundamental building blocks that it must have: a saliency system that gives proper attention to the important visible objects present in the current frame, a system that implements short-term memory, obtaining from them the important information and temporal correlations, a system for long-term memory that takes into account what happens in the remote past and, finally, a decision-making system that takes into consideration all this information.

%Determining an anomalous scene implies to evaluate the salient regions
Human's capability of evaluating danger on the fly is still a matter of study by neuroscientists.
When comparing machine to humans in the task of describing the content of an image we know that the latters perform better \cite{jiang2015salicon}.
Since we are aiming to develop a system capable of estimating danger in traffic video scenes, it is worth emulating how humans focus on important regions or objects in images through saliency estimation.
The commonly adopted saliency definition is based on how pixels/regions stand out and is dependent of what kind of visual stimuli human respond to the most \cite{yan2013hierachical}.
Cornia et al.\cite{cornia2016saliency} proposed Multi-Level Network for Saliency Prediction (MLNET), an architecture for saliency estimation which simulates what human see at first glance, a step forward towards driving scene comprehension.

%TODO Breve accenno su tranformer
Transformer have become the most popular approach in computer vision tasks, challenging the predominance of CNN architectures after the introduction of Vision Transformer (Dosovitskiy et al. \cite{DBLP:conf/iclr/DosovitskiyB0WZ21}).
Later, Liu et al. proposed Swin Transformer \cite{liu2021Swin}, keeping linear computational complexity to image size and builds hierarchical feature maps, allowing for a significant enhance in modelling power while keeping efficiency in latency regards.
Later on, the same authors designed Video Swin Transfomer \cite{liu_video_2022} as extension of Swin Transformer model but for video recognition tasks, by taking advantage of spatio-temporal locality of videos.
This model has proved to be very useful for building an efficient short-term memory, as it is capable of processing the last available frames in a single step.

\lnote{manca una frase di collegamento}

%Contribution
The main contributions of this paper are the following:
\begin{itemize}
    \item the enrichment of the information available to the final classifier with a compact latent features generated by a saliency model \cite{cornia2016deep};
    \item the adapting of the Video Swin Transformer model \cite{liu_video_2022} as a short-term memory encoder;
    \item the injection of a LSTM module inside the classifier head, to model the long-term memory of the past;
    \item an exhaustive ablation study on all the components of the architecture;
    \item a novel architecture called \lnote{XXX} to work in a real-time and online fashion for the online video anomaly detection task, which incorporates all previously defined modules.
\end{itemize}

The paper is organized as follows.
Section \ref{sec:related}, related work are introduced.
Section \ref{sec:theory}, the overall proposed architecture and an detailed in-depth analysis of all the components employed are shown.
Section \ref{sec:experiments}, a complete ablation studies and comparison with state-of-the-art are presented.
Section \ref{sec:conclusions}, final conclusions are drawn.
%\section{Related}
\label{sec:related}

\noindent\textbf{Vision Transformers}
Transformers \cite{vaswani2017attention} are born as an architecture to solve sequence-to-sequence problems, handling long-range dependencies in a simple way with the advantage of a strong parallelization compared to state-of-the-art architectures such as RNN and derivatives.
Initially developed for text analysis tasks, transformers have also found application in the image field.
The seminal work \cite{DBLP:conf/iclr/DosovitskiyB0WZ21} first proposed a Vision Transformer (ViT), paving the way for a new generation of detectors, alternative to CNN.
Afterwards, with the aim of improving the performance in terms of accuracy of the results and decreasing the computational need, variants such as the Swin Tranformers \cite{liu2021Swin} were born.
In order to reduce the computational cost of the self-attention mechanism, authors proposed a shifted-windowing scheme to compute self-attention on smaller non-overlapping windows, introducing cross-window connection to cope with the lack of connections between different regions of the image.
As a direct evolution of Swin Tranformers, to process video instead of images, a new architecture was proposed in \cite{liu_video_2022}.
The authors proposed to approximate spatiotemporal self-attention by compute self-attention locally, extending spatial domain to the spatiotemporal domain.

\noindent\textbf{Traffic Anomaly Detection}
To detect anomaly in video, in \cite{hasan2016learning}, authors proposed a convolutional AutoEncoder (ConvAE) trained only on normal frames with the objective of frames reconstruction.
In \cite{luo2017remembering, wang2018abnormal}, authors used Convolutional LSTM Auto-Encoder as framework to encode appearance and motion.
Authors of AnoPred \cite{liu2018future} proposed a multi-task loss which include image intensity, optical flow, gradient, and adversarial losses for video frame-level anomaly detection which apply a UNet to predict a future RGB frame.
In \cite{yao2019unsupervised}, authors proposed an unsupervised method which tracks traffic participants trajectories and detect anomaly monitoring prediction consistency.
In TRN model \cite{xu2019temporal}, authors coupled the action detection task with the future action anticipation during the training.
To predict the action, they use both the historical temporal dependencies modeled by a RNN and the anticipation of the future via a temporal decoder.
In 2020, authors of \cite{9712446} proposed a new dataset of video anomaly detection called Detection of Traffic Anomaly (DoTA).
% TODO check paper https://paperswithcode.com/paper/an-attention-guided-multistream-feature

% Online Action Detection
% Temporal Recurrent Networks for Online Action Detection
% Long Short-Term Transformer for Online Action Detection

\section{Memory-augmented Online VAD}
\label{sec:theory}

\fboxsep=1mm%padding thickness
\fboxrule=1pt%border thickness

\begin{figure}[!t]
            \centerline{\includegraphics[clip, width=\linewidth]{images/arch-rx-cropped.pdf}}
        \caption{The online frame-level Video Anomaly Detection (VAD) architecture. $f[t]$ is the frame at time $t$, $x$ the output of the Reducer, $\mathit{NF}$ the number of frames in input to the VST, $s[t]$ the anomaly classification score of the frame $f[t]$.\label{fig:arch}}
\end{figure}

In this section the MOVAD architecture is described.  
The model is composed by: a short-term memory module and a classification Head that includes a long-term memory module (see Fig.~\ref{fig:arch}). 
Taking inspiration from~\cite{xu2021long}, recently observed frames have been taken into account as a source of information related to the ongoing action, and past frames to take into account the context.

\noindent\textbf{Short-term memory module.}
Since we are dealing with the online version of VAD task, the only information available to the system, at any given time, are the current and the past frames.
In order to implement this module, we selected the Video Swin Transformer (VST)~\cite{liu_video_2022} over ViViT~\cite{Arnab_2021_ICCV} due to its superior performance, and over an RNN given its ability to process frames in parallel.
Originally born to carry out the Video Action Classification task, analyzing all the frames in one step, we adapted it to perform single-frame classification using few of them.
In particular, it considers only a small temporal window of $\mathit{NF}$ frames of the video, going from the current frame at time $t$ to the previous frame at time $t-\left(\mathit{NF}-1\right)$.
VST takes as input a video with size $\mathit{NF} \times H \times W \times 3$, where $\mathit{NF}$ ($=4$), $H$ e $W$ correspond to the number of frames, height, width and RGB channels, respectively.
The model internally splits the frames in non-overlapping 3D patches, partitioning the video in $\frac{\mathit{NF}}{2} \times \frac{H}{4} \times \frac{W}{4}$ 3D tokens, projecting the features to an arbitrary dimension $C$.
The rest of the VST architecture is similar to the original Swin Transformer~\cite{liu2021Swin}, with four stages of Video Swin Transformer blocks, interspersed with $2\times$ spatial down-sampling in the patch merging layer.

\noindent\textbf{Long-term memory module.}
The output of VST goes through Adaptive Average Pool 3D layer (Reducer in Fig.~\ref{fig:arch}) and, finally, enter inside the classification Head.
As shown in Fig.~\ref{fig:arch}, the Head is composed by a series of normalization layers, linear layers and dropout, alternating. 
It deals with long-term memory thanks to an LSTM module inserted after the last normalization layer.
This module is composed by three cells stacked together to form a stacked LSTM.
The state, composed by three hidden states $h[t]$ and three cell states $c[t]$, is updated whenever a new frame is available.
The LSTM receives in input a features block of $[B, 1024]$, where $B$ is the batch size, and returns a block of same size together with the state of the cells.
Because the state is relatively small, the module is very efficient and leads to a fixed and limited additional computational cost.
For each frame $f[t]$, the model outputs the anomaly classification score $s[t] \in [0,1]$, where $0$ means no anomaly and $1$ means the frame is anomalous.
A weighted cross-entropy loss was chosen to train the model, giving higher weight to the anomaly class, in order to reflect the distribution of the data.

% copyright image frame: <a href="https://www.freepik.com/free-vector/realistic-vector-icon-film-tape-strip-with-white-square-isolated-white-cinema-concept_31096470.htm#query=video%20frame&position=31&from_view=keyword">Image by user15245033</a> on Freepik
\section{Experiments}
\label{sec:experiments}

\noindent\textbf{Dataset}.
We perform our tests on the Dota dataset \cite{9712446}.
It contains 4677 videos taken from YouTube channels, with a resolution of $1024 \times 720$, annotated with information about the start and end of the anomaly, the category (10 in total) and the bounding boxes of the objects or persons involved.
The videos were recorded in different countries and with different light and weather conditions.
The dataset is split in approximately $70\%$ training and $30\%$ validation.
\lnote{li vogliamo ignorare o no? -> V: è un rischio perché andiamo abbastanza bene in quei casi, non saprei}
Because our benchmarks are related to the Task 1 \cite{9712446}, (online) frame-level Video Anomaly Detection, we ignore videos with unknown category or without objects, resulting in 1,305 test videos.
Furthermore, VO and OO columns are not shown because they do not contain anomalous traffic participants.

\noindent\textbf{Evaluation Metrics}.
To evaluate the performance of the models, we use the well-known Area Under Curve (AUC) metric.
This metric evaluate how well the model temporary-locate the anomaly in the videos.

\noindent\textbf{Implementation details.}
The results of the models with which we compare our model, are taken from the respective papers.
We perform the training on a single machine with 1 \lnote{XXX} GPU with \lnote{xxGB} of memory.
We use the Stochastic Gradient Descent (SGD) optimization algorithm with a learning rate of 0.0001, a \lnote{weight decay of 0.0001}, and a momentum of 0.9.

\noindent\textbf{Training details.}
Because the videos contain a non-uniform number of frames, to be able to fast training with batch-size major then one, we fixed the number of frames for each video taken into account.
In addition, we chosen online the starting frame for each video for each iteration, in a way to offer to the network as diverse as possible training and reduce the effect of overfitting.
\lnote{descrivi il modello pretrained su smthv2 e saliency pretrained}

\subsection{Ablation study}

% class weight loss
% 2xsoftmax vs 1x
% learning rate differences (sto finendo l’esperimento con multipli lr)

\noindent\textbf{Short-term memory module.}
% numero di frames in input
% input shape

\noindent\textbf{Long-term memory module.}
% posizione dell'lstm senza / prima / dopo / prim + dopo
% 1/2/.. # celle lstm

\noindent\textbf{Saliency module.}
% con/senza/versione ridotta della saliency

\noindent\textbf{Video clip length.}
% random_batch 4/8/12/16/20/24

\noindent\textbf{Model pretraining}
% training from scratch vs pretrained (imagenet vs smth2smthv2)

\noindent\textbf{XXXX model}
% versione finale vs resto del mondo su dota, and: 
%   - Phantom: https://paperswithcode.com/paper/approaches-toward-physical-and-general-video
%   - ShanghaiTech: https://paperswithcode.com/sota/anomaly-detection-on-shanghaitech
%   - CUHK Avenue: https://paperswithcode.com/sota/anomaly-detection-on-chuk-avenue
%   - UCSD Ped2: https://paperswithcode.com/sota/abnormal-event-detection-in-video-on-ucsd

\section{Conclusions}
\label{sec:conclusions}

In this paper, we propose a new architecture called MOVAD for the online frame-level Video Anomaly Detection (VAD) task.
It is capable to work in a real time guaranteed response times and online fashion, handling the most restrictive VAD scenario.
MOVAD is composed by a short-term memory, which extracts information related to the ongoing action, implemented by a Video Swin Transformer adapted to work in a online scenario, and a long-term memory module that considers remote past information thanks to LSTM.
We evaluated its performance on Detection of Traffic Anomaly (DoTA) dataset, a collection of dash-mounted camera videos of accidents, reaching $82.11\%$ AUC, surpassing the current state of the art by +$2.81$ AUC.

%\section*{Acknowledgments}
%\noindent This research benefits from the HPC (High Performance Computing) facility of the University of Parma, Italy. \mnote{roadster non ha vincoli di citazione?}

%\vfill\pagebreak

%\section{REFERENCES}
%\label{sec:refs}

% References should be produced using the bibtex program from suitable
% BiBTeX files (here: strings, refs, manuals). The IEEEbib.bst bibliography
% style file from IEEE produces unsorted bibliography list.
% -------------------------------------------------------------------------
{
\bibliographystyle{IEEEbib}
\bibliography{biblio}
%\bibliography{strings,refs}
}
\end{document}
