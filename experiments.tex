\section{Experiments}

\noindent\textbf{Dataset}.
We perform our tests on the Dota dataset \cite{9712446}.
The dataset contains 4677 videos taken from YouTube channels, with a resolution of $1024 \times 720$, annotated with information about the start and end of the anomaly, the category (10 in total) and the bboxes which indicate which objects or persons are involved.
The videos were recorded in different countries and with different light and weather conditions.
The dataset is split in approximately $70\%$ training and $30\%$ validation.
Because our benchmarks are related to the Task 1 \cite{9712446}, (online) frame-level Video Anomaly Detection, we ignore videos with unknown category or without objects, resulting in 1,305 test videos.
Furthermore, VO and OO columns are not shown because they do not contain anomalous traffic participants.

\noindent\textbf{Evaluation Metrics}.
To evaluate the performance of the models, we use the well-known Area Under Curve (AUC) metric.
This metric evaluate how well the model temporary-locate the anomaly in the videos.

\noindent\textbf{Implementation details.}
The results of the models with which we compare, are taken from the respective papers.
We perform the training on a single machine with 1 \lnote{XXX} GPU with \lnote{xxGB} of memory.
We use the Stochastic Gradient Descent (SGD) optimization algorithm with a learning rate of 0.0001, a \lnote{weight decay of 0.0001}, and a momentum of 0.9.

\subsection{Ablation study}