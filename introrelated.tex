\section{Introduction and related works}

Autonomous Vehicles (AVs) are becoming every day a reality thanks to the recent enormous scientific and technical advances.
Nevertheless, safety of AVs is still a relevant issue which can jeopardize their world-wide diffusion.
Increasing safety of AVs can be reached by providing vehicles with the ability of detecting anomalous situations in a prompt way.
Their detection provides information to avoid collisions, protect pedestrians, or re-route the current travel \cite{4298901}.
Among the different sensors exploited in AVs, cameras can provide rich real-time information about the scene.
Despite all, anomalous situation detection in real traffic scenarios is still very challenging, due to several reasons.
First of all, it requires real-time performance in order to allow the driver and/or the vehicle's actuators to react promptly.
Real-time performance does not only call for efficient algorithms, but also implies guaranteed response times and an online strategy, meaning that the system needs to rely only on current frames and, at most, knowing the past (or part of it), but w/out idea of future.
Secondly, we still miss a formal shared model of what an anomaly should be, because many times the riskness is highly subjective.
For instance, improper crossing of driving lanes might be perceived as anomalous and dangerous if long-lasting and in crowded traffic, while being negligible in other cases.
Thirdly, there are plenty of possible accident classes to take into account compared to normal traffic situations, and the number of available examples for some of them are very exiguous.
Lastly, the definition of precise time boundaries of an anomaly is even more subjective and doubtful.
Nevertheless, some attempts have been made to propose a deterministic method in the interest of defining an anomaly.
Fang \emph{et al.} \cite{fang2019dada} want to predict an accident likely to happen in the next 5 seconds, labelling the anomaly start from the moment in which half part of the object involved in the accident appears in the view.
Yao \emph{et al.}~\cite{9712446} proposed the Detection of Traffic Anomaly (DoTA) dataset, that takes into account when the anomalous event starts and ends, spatially locating the entities involved and classify the type.
They formulate the anomaly start as the instant after which the accident is unavoidable (still, a subjective concept).
We takes this last dataset as the benchmark for our work.

Despite these premises, there have been several previous works addressing the problem of traffic anomaly detection.
Authors in \cite{hasan2016learning} proposed a Convolutional AutoEncoder (ConvAE) trained only on normal frames with the objective of frame reconstruction.
In \cite{luo2017remembering, wang2018abnormal}, authors used Convolutional LSTM Auto-Encoder as framework to encode appearance and motion.
As noted by \cite{ramachandra2020survey}, auto-encoder-type reconstruction methods are sensitive to the amount of anomalies that occur in the scene and many times they require additional post-processing techniques.
Authors in \cite{liu2018future} proposed AnoPred, which uses a multi-task loss, including image intensity, optical flow, gradient and adversarial losses for Video Anomaly Detection (VAD) by predicting a future frame.
As stated in~\cite{9712446}, AnoPred was thought in a video surveillance context, while videos acquired from a moving vehicle are more dynamic and difficult to predict.
In \cite{zhou_spatio-temporal_2022}, authors make coarse-to-fine detection, by encoding the temporal features with Histogram of Optical Flow (HOF) \cite{wang2013action} and ordinal features of frames by a CNN, and then by encoding the CNN features and spatial relationships of the objects.
In FOL~\cite{9712446}, authors try to avoid the future prediction for the entire frame, focusing instead on tracking actors position and predicting their future locations.
Conversely to us, both methods are limited by the presence of actors in the Field of View (FOV).
In addition, our model requires a lot less information (only RGB frames) than AnoPred~\cite{liu2018future}, FOL \cite{9712446} and STFE \cite{zhou_spatio-temporal_2022}.
In \cite{xu2019temporal}, the TRN model couples the action detection task with both the temporal dependencies modeled by a RNN and the anticipation of the future via a temporal decoder.
In our case, we do not speculate on the future, but rely only on the present and the past.

In this paper, we describe our new architecture called \emph{MOVAD} (Memory-augmented Online Video Anomaly Detection).
In order to build a system that is capable to temporally locate anomalies in videos, belonging to a wide range of possible situations, we have introduced two main architectural contributions.
The first exploits a Video Swin Transformer~\cite{liu_video_2022} (VST) model, adapted to model a short-term memory in a real-time and online scenario.
This adaptation allows the system to incorporate temporal correlation and to detect online behaviors, by relying only on the current and few previous frames.
The second contribution is the injection of a LSTM module inside the classification Head to model the long-term memory of the past.
This is crucial to exploit contextual information spread into the entire video.
An exhaustive ablation study is done over the dataset DoTA \cite{9712446}.
Compared w.r.t.~the state of the art, MOVAD shows superior performance in terms of AUC.
It is worth noting that our model works in a more restrictive scenario than most of other models, adding a further degree of difficulty given by the real time guaranteed response times and online capabilities.