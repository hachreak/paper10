\section{Introduction and related works}
Autonomous Vehicles (AVs) are becoming every day a reality thanks to the enourmous scientific and technical advances. Nevertheless, safety of AVs is still a relevant issue which can jeopardize their world-wide diffusion. Increasing safety of AVs can be reached by providing vehicles with the ability of detecting accidents in a prompt way. Even more, also anomalous traffic situations are relevant for many automatic tasks of AVs. Their detection provides information to avoid collisions, protect pedestrians, or re-route the current travel \cite{4298901}. Moreover, if a V2V network infrastructure (such as VANET \cite{fatemidokht2021efficient}) is available, dissemination of useful information can provide other connected vehicles insights about the traffic status. 

Among the different sensors equipped in AVs, cameras can provide rich real-time information about the scene. Real-time video analysis can make useful tools available for driving scene interpretation in the context of Advanced Driver Assistance Systems (ADAS). Despite the gigantic advances of video analysis algorithms, anomalous situation detection in real traffic scenarios is still very challenging, due to several reasons.

First of all, detection of traffic anomalies requires real-time performance in order to allow the driver and/or the vehicle's actuators to react promptly. Real-time performance does not only call for efficient algorithms, but also implies an online strategy, meaning that the system needs to rely only on current frames and, at most, knowing the past (or part of it), but having no idea of what happens in the future. Secondly, aiming at detecting anomalies in traffic scenes, a proper, formal and generally-acknowledged model of what an anomaly is should be available. Unfortunately, this shared model does not exist and, in any case, would be likely to be subjective. For instance, improper crossing of driving lanes might be perceived as anomalous and dangerous if long-lasting and in crowded traffic, while being negligible in other cases. In addition, there are plenty of possible accident classes that must be taken into account compared to normal traffic situations, and different classes might require different actions.

Lastly, in a (semi-)supervised machine learning setting and aiming at the precise localization in time of the anomaly, not only definition and type (class) of anomalies matter, but also their starting and ending instants/frames. Definition of precise boundaries of an anomaly/accident is even more subjective and doubtful. We can define an anomaly as an hazardous situation that can lead to an accident, but, since the hazardousness prior to the accident may be determined subjectively by each individual, the time interval edges for an anomaly are not really clear and this is reflected in dataset labelling.

Nevertheless. some attempts have been made to propose a deterministic method in the interest of defining an anomaly.
%Yao et al. \cite{yao2019unsupervised} defines an anomaly as the window in which the accident happens, but since we want to prevent it, this might not be ideal in a prevention perspective. \lnote{pero' noi non facciamo prevention, quindi?}
First, Fang \emph{et al.} \cite{fang2019dada} want to predict an accident likely to happen in the next 5 seconds, labelling the anomaly start from the moment in which half part of the object involved in the accident appears in the view.
Yao \emph{et al.} \cite{yao2020when} proposed a Detection of Traffic Anomaly (DoTA) dataset that takes into account when the anomalous event starts and ends, locates spatially where all the involved entities are in each frame and what type of anomaly is. Their work formulates the anomaly start as the instant after which the accident is unavoidable (still, a subjective concept).

Despite all these premises, there have been several previous works addressing the problem of traffic anomaly detection. Authors in \cite{hasan2016learning} proposed a convolutional AutoEncoder (ConvAE) trained only on normal frames with the objective of frames reconstruction.
In \cite{luo2017remembering, wang2018abnormal}, authors used Convolutional LSTM Auto-Encoder as framework to encode appearance and motion. The paper in \cite{liu2018future} proposed 
AnoPred which uses a multi-task loss including image intensity, optical flow, gradient, and adversarial losses for video frame-level anomaly detection by applying a UNet to predict a future RGB frame.
%In \cite{yao2019unsupervised}, authors proposed an unsupervised method which tracks traffic participants trajectories and detect anomaly monitoring prediction consistency.
Considering also to predict future situations in the scene, the TRN model \cite{xu2019temporal} coupled the action detection task with both the historical temporal dependencies modeled by a RNN and the anticipation of the future via a temporal decoder. \anote{Manca commento critico ai related}
%In 2020, authors of \cite{9712446} proposed a new dataset of video anomaly detection called Detection of Traffic Anomaly (DoTA).

In this paper, in order to build a system that is capable of recognizing different types of anomalous situations and locate them temporally in the video, we have introduced two main architectural contributions. The first exploits a Video Swin Transformer (VST) network \cite{liu_video_2022}, adapted to model short-term memory. This adaptation allows the system to incorporate temporal correlation and to reproduce online behaviors, by relying only on the current and few previous frames.

Transformers \cite{vaswani2017attention} have been proposed as an architecture to solve sequence-to-sequence problems, handling long-range dependencies in a simple way with the advantage of a strong parallelization compared to state-of-the-art architectures such as RNNs and variants.
Initially developed for text analysis tasks, transformers have also found application in the image field.
The seminal work \cite{DBLP:conf/iclr/DosovitskiyB0WZ21} first proposed a Vision Transformer (ViT), paving the way for a new generation of detectors, alternative to CNN.
Afterwards, with the aim of improving the performance in terms of accuracy of the results and decreasing the computational need, variants such as the Swin Tranformers \cite{liu2021Swin} were proposed. In order to reduce the computational cost of the self-attention mechanism, authors proposed a shifted-windowing scheme to compute self-attention on smaller non-overlapping windows, introducing cross-window connections to cope with the lack of connections between different regions of the image.
As a direct evolution of Swin Tranformers, to process video instead of images, a new architecture was proposed in \cite{liu_video_2022}. The authors proposed to approximate spatio-temporal self-attention by compute self-attention locally, extending spatial domain to the spatio-temporal domain. \anote{anche qui qualcosa su come ci differenziamo}

The second architectural contribution is the injection of a LSTM module inside the classifier head to model the long-term memory of the past. This is crucial to have the network learning from the past situations how to better model anomalies. Other contributions of this paper include an exhaustive ablation study on all the components of the architecture, called \emph{MOVAD}, and the extensive experiments over a public dataset, showing superior performance in terms of AUC wrt the state of the art, by also incorporating real-time and online capabilities. 